\section{Service networks}
\label{sec:networks}

In this section we describe how the tools for peer-to-peer accounting and data exchange can be used to drive the incentives for distributed digital services. Assume that a set of nodes form a network that provides a decentralised service. Insured chunk storage in swarm, arbitrary remote payments or database insertion are examples of such services. 
A \gloss{service task} is defined as an instance of service provision. Storing a particular chunk, paying an amount to another node or inserting an entry to an index are examples of tasks of the respective network service.

In what follows we will work with the assumption that the participating nodes in a service network are connected in a \gloss{kademlia topology} and can relay messages to other nodes using kademlia deterministic routing based on direct devp2p transport layer for each hop. In other words service networks are composed of a subset of nodes within swarm.

Scalability of swap channels is based on the premise that directly connected nodes engage in long-term repeated dealings and can afford setting up contracts on the blockchain to secure their interaction.%
%
\footnote{Given the semi-permanent connections of the TCP based kademlia of swarm, peers interact with $O(log(n))$ peers where $n$ is the number of nodes in the network.}
%
A \gloss{swap channel network} enables \gloss{indirect transactions} by relaying tasks and deliveries using swap-channel transactions on every hop. This makes it possible to preserve the scalability and security of swap, yet extend the scope of transactions both in terms of frequency and reach, i.e., enabling \emph{ad-hoc one-off interactions between any two non-connected nodes}. 
Service networks with global provision guarantees further improve the scope by offering guaranteed market making and delivery in a direct immediately settling swap transaction. As a result non-participating users can just \emph{request and disappear}.

\subsection{Incentivisation for relaying}

Let's assume that nodes A,B,C,D are participant peers of a service network such that
A-B, B-C and C-D are direct connections with a swap channel.
Assume further that A intends to do some ad-hoc one-off business with D. To this end,  A issues a conditional bond specifying an escrow condition that some data be obtained from D constituting the proof of delivery of the task.

When A sends the conditional bond to B as beneficiary, B receives the conditional bond and issues the same bond only this time with C as beneficiary. C is directly connected to D, so it just relays the same to D who issues a receipt (or takeover proof). 
Each of these steps is implemented as swaps of conditional notes. The moment C receives the receipt, it is incentivised to pass it back to B as an invoice to prompt settlement. The same is true for B and any number of intermediate relaying nodes (see figure \ref{fig:svcnets-sequence-diagram}).

% \PreviewEnvironment{tikzpicture}
% \setlength\PreviewBorder{7pt}
\definecolor{pinegreen}{cmyk}{0.92,0,0.59,0.25}
\definecolor{royalblue}{cmyk}{1,0.50,0,0}
\definecolor{violet}{cmyk}{0.79,0.88,0,0}
\tikzstyle{cblue}=[circle, draw, thin,fill=cyan!20, scale=0.8]
\tikzstyle{qgre}=[rectangle, draw, thin,fill=green!20, scale=0.8]
\tikzstyle{rpath}=[thin, densely dotted, black, opacity=0.7]
\tikzstyle{gpath}=[thick, pinegreen, opacity=0.4]
\tikzstyle{dpath}=[ultra thin, black, opacity=0.4]
\tikzstyle{legend_isps}=[rectangle, rounded corners, thin, 
                       fill=gray!20, text=blue, draw]
                        
\tikzstyle{legend_overlay}=[rectangle, rounded corners, thin,
                           minimum width=2.5cm, minimum height=0.8cm,
                           pinegreen]
\tikzstyle{legend_phytop}=[rectangle, rounded corners, thin,
                          minimum width=2.5cm, minimum height=0.8cm,
                          royalblue]
\tikzstyle{legend_general}=[rectangle, rounded corners, thin,
                          minimum width=2.5cm, minimum height=0.8cm,
                          violet]
\tikzstyle{node_label}=[rectangle, thin,
                          minimum width=0.2cm, minimum height=0.2cm,
                          black]
\tikzstyle{node_legend}=[rectangle, thin,
                          minimum width=1.2cm, minimum height=0.2cm,
                          black]

% Diagram
\begin{figure}
\centering
   

\begin{tikzpicture}[auto, thick]
 % Cloud creation
  \node[cloud, fill=gray!20, cloud puffs=16, cloud puff arc= 100,
        minimum width=7cm, minimum height=2.5cm, aspect=1] at (0,0) {};

 % Physical layer nodes
  \foreach \place/\x in {{(-2.5,0.3)/1}, {(-1.75,-0.55)/2},{(-1.2,0.55)/3},
    {(-0.75,-0.7)/4}, {(-0.25,0)/5}, {(0.25,0.7)/6}, {(0.75,-0.3)/7}, 
    {(1.5,0)/8},{(2.5,0.4)/9}}
  \node[cblue] (a\x) at \place {};
 
 % Physical layer links
  \path[thin] (a1) edge (a2);
  \path[thin] (a1) edge (a3);
  \path[thin] (a2) edge (a3);
  \path[thin] (a3) edge (a6);
  \path[thin] (a2) edge (a4);
  \path[thin] (a5) edge (a6);
  \path[thin] (a5) edge (a4);
  \path[thin] (a5) edge (a2);
  \path[thin] (a5) edge (a7);
  \path[thin] (a6) edge (a7);
  \path[thin] (a6) edge (a9);
  \path[thin] (a6) edge (a8);
  \path[thin] (a8) edge (a9);
  \path[thin] (a7) edge (a8);
 


\node[node_label] (A) at (-2.5,2.6) {A};
\node[node_label] (B) at (-1.2,2.85) {B};
\node[node_label] (C) at (0.75,1.6) {C};
\node[node_label] (D) at (2.25,2.3) {D};

%swap connection legend nodes
\node (LS) at (4.2,1.4) {};
\node (LE) at (5.2,1.4) {};
\path[gpath] (LS) edge (LE);

 % Overlay layer nodes
  \foreach \place/\i in {{(-2.5,2.3)/1},{(-1.2,2.55)/2},
    {(0.75,1.3)/3}, {(2.25,2)/4}}
    \node[qgre] (b\i) at \place {};
 
 % Overlay layer links
  \path[dpath] (b1) edge (b2);
 % \path[dpath] (b1) edge (b4);
 % \path[dpath] (b2) edge (b4);
  \path[dpath] (b4) edge (b3);
 % \path[dpath] (b3) edge (b1);

  \path[gpath] (b1) edge (b2);
  \path[gpath] (b2) edge (b3);
  \path[gpath] (b3) edge (b4);

 % Links between overlay and physical topology
  %\foreach \i in {1,...,4}
  %  \path[rpath] (a\i) edge (b\i);
  \path[rpath] (a1) edge (b1);
  \path[rpath] (a3) edge (b2);
  \path[rpath] (a7) edge (b3);
  \path[rpath] (a9) edge (b4);
 % Legends
  \node[legend_general] at (0,4){\textsc{Swarm Service Networks}};
  \node[legend_overlay] at (6,2){\textsc{Kademlia Overlay}};
  \node[legend_phytop] at (6,0){\textsc{Physical Network}};
  \node[node_legend, font=\fontsize{4}{4.5}\selectfont] at (6.2,1.4) {Direct Swap Connection};

 
 
\end{tikzpicture}
\caption{Swap channel network: peer to peer network of nodes with semipermanent TCP connections organised in
a kademlia topology. Given nodes have an associated identity owning a chequebook contract,
each live connection defines a swap channel.}
\label{fig:svcnets-overlay-diagram}
\end{figure}

% Agents
\def\A{A}
\def\B{B}
\def\C{C}
\def\D{D}

% Message Flows
\def\BondB{Conditional Bond A->B}
\def\BondC{Conditional Bond B->C}
\def\BondD{Conditional Bond C->D}
\def\ReceiptD{Receipt D->C}
\def\ReceiptC{Receipt C->B}
\def\ReceiptB{Receipt B->A}

% Legend 
%\def\LegendOnChain{On-chain}
%\def\LegendOffChain{Off-chain}

% Diagram
\begin{figure}
\centering
   

% Diagram
\begin{tikzpicture}[every node/.style={font=\normalsize,
  minimum height=.75cm,minimum width=1cm},]

% Matrix
\node [matrix, very thin,column sep=1.0cm,row sep=0.2cm] (matrix) at (0,0) {
  & \node(0,0) (\A) {};   &                         & \node(0,0) (\B) {};   &                         & \node(0,0) (\C) {};     &                         & \node(0,0) (\D) {}; \\
  & & & & & & & \\
  & & & & & & & \\
  & & & & & & & \\
  & \node(0,0) (\A 1) {}; & \node(0,0) (\BondB) {}; & \node(0,0) (\B 1) {}; &                         & \node(0,0) (\C 1) {};   &                         & \node(0,0) (\D 1) {}; \\
  & & & & & & & \\
  & & & & & & & \\
  & \node(0,0) (\A 2) {}; &                         & \node(0,0) (\B 2) {}; & \node(0,0) (\BondC) {}; & \node(0,0) (\C 2) {};   &                         & \node(0,0) (\D 2) {}; \\ 
  & & & & & & & \\
  & & & & & & & \\
  & \node(0,0) (\A 3) {}; &                         & \node(0,0) (\B 3) {}; &                         & \node(0,0) (\C 3) {};   &  \node(0,0) (\BondD) {};& \node(0,0) (\D 3) {}; \\
  & & & & & & & \\
  & & & & & & & \\
  & \node(0,0) (\A 4) {}; &                         & \node(0,0) (\B 4) {}; &                         & \node(0,0) (\C 4) {};   &  \node(0,0) (\ReceiptD) {}; & \node(0,0) (\D 4) {}; \\
  & & & & & & & \\
  & & & & & & & \\
  & \node(0,0) (\A 5) {}; &                         & \node(0,0) (\B 5) {}; & \node(0,0) (\ReceiptC) {};& \node(0,0) (\C 5) {}; &                          & \node(0,0) (\D 5) {}; \\
  & & & & & & & \\
  & & & & & & & \\
  & \node(0,0) (\A 6) {}; & \node(0,0) (\ReceiptB) {};& \node(0,0) (\B 6) {}; &                         & \node(0,0) (\C 6) {};&                          & \node(0,0) (\D 6) {}; \\
  & & & & & & & \\
  & & & & & & & \\
  & \node(0,0) (\A 7) {}; &                         & \node(0,0) (\B 7) {};  &                          & \node(0,0) (\C 7) {};&                          & \node(0,0) (\D 7) {}; \\
  & & & & & & & \\
  & & & & & & & \\
%  & \node(0,0) (\A 7) {}; & \node(0,0) (\LegendOnChain) {};  & & & & & \\
%  & \node(0,0) (\A 8) {}; & \node(0,0) (\LegendOffChain) {}; & & & & & \\
};

% Agents labels
\fill 
	(\A) node[draw,fill=white] {\A}
	(\B) node[draw,fill=white] {\B}
	(\C) node[draw,fill=white] {\C}
	(\D) node[draw,fill=white] {\D};

\draw [dashed] 
  (\A) -- (\A 7)
  (\B) -- (\B 7)
  (\C) -- (\C 7)
  (\D) -- (\D 7);

% Horizontal flows (Monetary interactions)
\draw [-{Latex[length=2mm,width=3mm]}] (\A 1) -- (\B 1);
\draw [-{Latex[length=2mm,width=3mm]}] (\B 2) -- (\C 2);
\draw [-{Latex[length=2mm,width=3mm]}] (\C 3) -- (\D 3);
\draw [-{Latex[length=2mm,width=3mm]}] (\D 4) -- (\C 4);
\draw [-{Latex[length=2mm,width=3mm]}] (\C 5) -- (\B 5);
\draw [-{Latex[length=2mm,width=3mm]}] (\B 6) -- (\A 6);

% Flows Labels 
\fill
  (\BondB) 
    node[above] {\BondB}
  (\BondC) 
    node[above] {\BondC}
  (\BondD) 
    node[above] {\BondD}
  (\ReceiptD) 
    node[above] {\ReceiptD}
  (\ReceiptC) 
    node[above] {\ReceiptC}
  (\ReceiptB) 
    node[above] {\ReceiptB};

% Interaction points 
\draw 
  (\A 1) node[minimum size=0.2cm, draw,circle,fill=red!20] {}
  (\B 1) node[minimum size=0.2cm, draw,circle,fill=red!20] {}
  (\B 2) node[minimum size=0.2cm, draw,circle,fill=red!20] {}
  (\C 2) node[minimum size=0.2cm, draw,circle,fill=red!20] {}
  (\C 3) node[minimum size=0.2cm, draw,circle,fill=red!20] {}
  (\D 3) node[minimum size=0.2cm, draw,circle,fill=red!20] {}
  (\D 4) node[minimum size=0.2cm, draw,circle,fill=red!20] {}
  (\C 4) node[minimum size=0.2cm, draw,circle,fill=red!20] {}
  (\C 5) node[minimum size=0.2cm, draw,circle,fill=red!20] {}
  (\B 5) node[minimum size=0.2cm, draw,circle,fill=red!20] {}
  (\B 6) node[minimum size=0.2cm, draw,circle,fill=red!20] {}
  (\A 6) node[minimum size=0.2cm, draw,circle,fill=red!20] {};

\end{tikzpicture}
\caption{Sequence of conditional bond exchange and receipt forwarding in a service network}
\label{fig:svcnets-sequence-diagram}
\end{figure}

In general, we define an \emph{indirect transaction} as a chain of swaps between directly connected
peers. The success of a such indirect transactions between two nodes is dependent on whether and
how such a chain can be found. Our assumption is that the nodes have semi-permanent connectivity
and form a network topology where routing between any two nodes is guaranteed, e.g., kademlia. As a result, we can assume that as long as all nodes on the chain have a healthy kademlia connectivity, the prerequisite for finding the route is satisfied.

In order to incentivise relaying nodes to take part in such a network, a transaction fee for every hop needs to be introduced.  We stipulate that the transaction fee for one swap is proportional to the logarithmic distance that the hop spans. As a consequence, the simple rational strategy to maximise profit will incentivise nodes to find the closest node when relaying as well as maintain a healthy connectivity needed for successful routing. Furthermore, the entire transaction fee can be precalculated as proportional to the distance between sender and remote beneficiary and therefore can be offered in advance when issuing the conditional bond.

This construct is equivalent to message relaying with \gloss{certified delivery}, except that nodes are required to have funds to cover the amount of the conditional bond.
For live connections in a service network to be operational, parties agree to always having in-channel capacity in the amount of X. With soft channel deposits, such throughput restrictions can easily be relaxed and adjusted even on an on-demand basis.

In addition to swap channels, connected nodes operate \gloss{provable data exchange}. Assume that two nodes participate in a particular service that involves relaying objects (such as requests and responses). The content addresses (swarm hash) of consecutive objects sent in one direction constitute a \emph{data stream}. Both peers save this stream to swarm and maintain an index mapping the hashes to offsets. The upstream peer (sender) calculates the swarm hash of this index and periodically signs it against the current blockheight or timestamp as well as the data stream identitfier. This \gloss{handover state} is periodically sent to the downstream peer (receiver)
who verifies it and preserves it until the next one. The downstream peer periodically countersigns the handover state together with an initial offset or timestamp. This \gloss{takeover state} is then sent to the upstream peer. For instance, the handover state obtained from the upstream peer can be used to prove to third parties that an object X was handed over to the downstream peer. This can be done by giving an inclusion proof of the hash of X.%
%
\footnote{Upstream peer can be challenged to give an inclusion proof.}

On top of the rewards peers get for relaying, we can introduce additional incentives to guarantee that messages reach the recipient. We can assume that relaying nodes register on a service contract they promise to relay messages in a data stream towards the recipient. If nodes have a stake to lose, punitive measures can be imposed on nodes that fail or refuse to relay messages in the data stream.

Take our earlier example of A sending a message to remote node D, and assume that B, C, and D are all registered relaying nodes that are online. If there is reason to believe that the message did not reach D, A can challenge B by simply providing the takeover proof for the message. B can defend itself by providing takeover proof from C, thereby shifting the blame to C for blocking the delivery. C in turn can show takeover proof from D to refute the challenge. It can also happen that C indeed did not forward the message because there was no node closer to D among its peers. But this means that D is a nearest neighbour of C, in which case D should have been connected to C.
Therefore D can be challenged to prove that it was connected to C. D can respond to the challenge
by showing handover state obtained from C dated after the message delivery deadline. This in turn is
a challenge to C to show the inclusion proof of the message against the handover state.
This pattern is called \gloss{finger pointing} and constitutes the investigative part
of litigation which results in identifying the node
ultimately responsible for failed delivery by not relaying.
Finger pointing essentially implements \emph{guaranteed delivery}.

\subsection{Uniform resource allocation and market making}

In the previous section we just assumed that A knew about D (from an independent source). In case there are other nodes as well that can in theory perform the same task, a bounty is sent to potentially multiple addresses of candidate providers and an implicit market making mechanism is obtained by certified delivery.

Assume the tasks of a service network can be provided by any participating node using the same amount of resources. Given a network of willing providers, a system can be designed called \gloss{global resource allocation by network topology (GRANT)} such that tasks are distributed more or less uniformly across the network.

For instance, this is the case with storage in swarm: each task of storage provision involves storing a fixed-size chunk of data. We index the chunks with their content address (swarm hash) and allocate the storage task to nodes whose address is closest to the chunk address. Since both nodes and chunks are uniformly distributed in the same address space, this strategy implements uniform load balancing. We can generalise this allocation strategy for any task if requirements are uniform across tasks by simply indexing the task requests with their hash.

When a service user issues a bounty to request a service task, the bounty is sent towards the task's address using kademlia routing. The node responsible for servicing the task is dynamically defined as the closest node at any point in time. If hops use provable data streams, the incentives can make sure that the request (task bounty note) reaches the appropriate nodes and thereby are granted the opportunity to deliver the task.

For continuous services such as chunk storage, this means that as nodes leave and join the network, the node responsible for a task can change over time. With data exchange proofs however, we can enforce that nodes synchronise historical task requests and therefore the allocation can be taken for granted. In other words, the network can perform dynamic load-balancing of previously submitted future tasks even though the network is changing.

In order to benefit from the allocated tasks by cashing the bounties on delivery, the distribution
should be incentivised. In particular relaying nodes are not supposed to withhold tasks in hope of cashing the bounty themselves. To guarantee that downstream peers do get forwarded all the tasks and bounties, upstream peers can be challenged when they cash a bounty: if there is a valid handover proof of connection (synchronisation) for the relevant period after the bond was issued, upstream peer can be challenged for not relaying it (but withholding). Conversely, the bounty outpayment can be challenged if the candidate provider is not the closest node to the task, i.e., a downstream peer closer to the task address can show connectivity during the relevant period. Withholding if proven could result in the same punitive measure as failing to deliver.


Service nodes that are capable of staying online can choose to register and put up a deposit as collateral. One consequence of GRANT is that for each task, there is a node (or set of neighbouring nodes) that is responsible for delivering the task and can ultimately be held liable for non-delivery. Using the swindle scheme for litigation, if proven guilty a node's collateral is forfeited. Since individual punitive measures are a very effective incentive, service guarantees can be given independent of the positive incentivisation (the reward for delivery implicit in the bounty).%
%
\footnote{Since the amount of tasks (and therefore capacity requirements) are roughly uniform for each participant node, catastrophic failure of an entire node can cause limited damage. Since the required deposit is meant to collateralise this risk, it is reasonable to set it to a fixed amount across all provider nodes.}

When the original issuer of the task request receives the receipt for its bounty (takeover proof) from its immediate peer, they have all that is needed to start finger pointing after the deadline for the service task passed. In other words, service users have \emph{instant guarantees} that the service will be provided for the task they request, even though it is unknown at the time by which node.

This type of service network is called \gloss{warranted automatic service provision (WASP)} network.
Since one can thus obtain instant enforceable service guarantees in one swap exchange, WASP networks allow users to \gloss{request and disappear}. Applying this scheme to swarm insured storage, given a WASP network of storage nodes, swarm users can \emph{upload and disappear}.

\subsection{Atomic swap}

The typical market making mechanism is the swap of request, offer, agreement. This pattern is described as a conditional note issued by A specifying an escrow that takes a conditional bond referring to the first as evidence. 
If this is countersigned by the original issuer, the testimony is positive, otherwise N/A.
This construction can be abstracted out as a generic connector witness which is equivalent to the logical AND operator. It validates two conditional notes and check their mutual commitment to binding them atomically. 

Another usecase is decentralised orderbooks