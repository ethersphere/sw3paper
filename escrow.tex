%\section{Escrow conditions on future payments}
\section{Promissory notes}

A promissory note is a redeemable promise of payment potentially maturing in the future. For example a cheque (see section \ref{subsubsec:per-user-guarantees}) is a promise of payment that can be redeemed at any time by `cashing' the cheque. 

We define the general notion of a promissory note as a \emph{promise} that entitles a \emph{beneficiary} to \emph{future payment} under a \emph{condition}. Since any such promise effectively puts the money in escrow, the condition can also be referred to as an  \gloss{escrow condition}. A promissory note can contain the following fields:

  \begin{itemize}
    \item an index (serial number)
    \item an amount
    \item a beneficiary (ethereum address)
    \item an escrow condition (escrow witness contract address)
    \item a valid-from time (blockheight or time)
    \item a valid-until time (blockheight or time)
    \item a remark
  \end{itemize}

In the simplest case in which no escrow condition or maturation date is given: the note represents a simple cheque as discussed above.
A maturation date for the promissory note (expressed as a blockheight or timestamp) can be specified to indicate the earliest possible occasion a note can be redeemed (valid-from) as well as a deadline when the promise expires (valid-until).
During the validity period, the promise can be redeemed only if the escrow condition is met. The active period of a note lasts from issuance to expiry.

\subsection{Recurring payments}

A cheque represents an adhoc promise to redeem an amount. A transfer of funds is sanctioned by the issuance of a cheque and cheques are issued as needed. There are, however, services where recurring payments are predictable and therefore the installments can be sanctioned in advance. The primary example of this are subscription based services. 

The chequebook contract can be expanded to support such services. Authorising a recurring payment (to a contract) would be achieved by signing a blank cheque (a cheque with an unspecified amount) against the beneficiary possibly with a valid-until date. The beneficiary (a smart contract) would then be able to withdraw payments from the chequebook at will. It would be up to the smart contract code of the beneficiary to ensure that only the `correct' amounts are transferred. 

\subsection{Bonds}

A note with an amount, a beneficiary and a future valid-from date is effectively a bond. The amount represents all outstanding payment obligations that the beneficiary is entitled to once the bond matures. Such a note can either be collateralised (as in section \ref{subsubsec:per-user-guarantees}) or it can come without a guarantee (as in \ref{subsec:simple-chequebook}). When such a note is not collateralised at the time of issuance, accepting the bond is equivalent to granting a loan: A sends funds to B and B issues an unsecured bond in return.


\subsection{Escrow conditions and escrow witnesses}

Conditions for payment that are more complex than a validity period are captured by the escrow condition. A promissory note can specify an address of an escrow contract which is to act as the judge, determining whether the escrow conditions have been met.

A note with an escrow field specified is called a \gloss{conditional bond} and can represent a \gloss{service request} with the escrow condition defining what constitutes the successful delivery of the service. 
If no beneficiary is specified, such note is essentially a \gloss{bounty} - a payment to be made to the first person who can satisfy the escrow condition. 
Bonds and bounties with escrow condition are subsumed under \gloss{conditional notes}.

Thus, in order for payments to be made, the escrow condition must be met. When the escrow condition is to be verified, the owner, the beneficiary address, and the note id (the hash of the signed note) are passed as arguments to the {\gloss{testimonyFor}} method of the escrow contract. By implementing this method the contract conforms to the \gloss{witness} contract  interface (section \ref{sec:courtroom}). Given the witness contract, the escrow condition is implicitly defined as whatever state makes the escrow witness give a positive testimony. 


Figure \ref{fig:taxonomy} summarises the various types of promissory notes.

\newcommand{\tick}{\checkmark}
\newcommand{\opt}{?}
\begin{center}
\begin{figure}
\begin{center}
\begin{tabular}{|l|r||c|c|c|c|c|c|c|}
\hline
note & fields
& index
& amount
& beneficiary
& escrow
& valid-from
& valid-until
& remark
\\
\cline{2-9}

type & type 
& int256
& int256
& address
& address
& int256
& int256
& byte32
\\
\hline
\hline
\multicolumn{2}{|l||}{cheque}   & \tick & \tick & \tick & & & \opt& \opt
\\
\multicolumn{2}{|l||}{authorisation} &  & \tick & \tick & & & \opt& \opt
\\
\multicolumn{2}{|l||}{bond} & \tick & \tick & \tick & & \tick & \opt& \opt
\\
\multicolumn{2}{|l||}{conditional bond} &  & \tick & \tick & \tick & \tick & \opt& \opt
\\
\multicolumn{2}{|l||}{bounty} & \tick &  \tick & & \tick & \tick & \opt& \opt
\\
\multicolumn{2}{|l||}{soft channel deposit} &  \tick & \tick & & & && \opt
\\
\hline
\end{tabular}
\end{center}
\caption{Taxonomy of promissory notes: '\tick' indicates a mandatory field, '?' indicates optional field. Types show the corresponding solidity type to encode in the ABI. }
\label{fig:taxonomy}
\end{figure}
\end{center}


% \subsection{Secure escrow and liquidity}

\subsection{Conditional bonds and Swap: invoice and cheque}

Let us assume that node A issues a conditional note (bond or bounty) to B with expiry (valid-until) time $T$. $ T$ represents the earliest time that A can consider the note unfulfilled (the service undelivered), before that we say that the note is \emph{active}. Let us further assume that B fulfills the condition while the note is active.

Invoking the escrow contract directly B can of course redeem the note, however it would be nice if this could be subsumed in the Swap traffic so that no expensive on-chain operation is necessary. 
To redeem the note in Swap, B notifies A of successful delivery of the service and issues  an \gloss{invoice}. This invoice contains the note id, the current cumulative swap balance and the serial number of the last cheque. The invoice thus contains the swap balance at the time of delivery as well as the note that is to be redeemed. It serves to indicate that the subsequent cheque will be the one to pay the outstanding amount for the note.

As a response, A is expected to send a cheque with an updated channel balance reflecting the payment of the invoice; i.e., the amount in the conditional note is added to the cumulative total. The remarks field references the conditional note id or the invoice id. If A refuses to do this, B can still send the conditional note in a transaction to the escrow contract on-chain. When the witness validates the delivery condition, and the testimony is positive, the amount is escrowed and a grace period starts. During the grace period, A can respond by sending in the appropriate cheque to the contract as proof that the cheque was issued.

Conditional notes correspond to service requests. If peer B accepts and decides to act on the conditional bond, it is prudent to require that total liability over all active bonds with B. If this total is higher than the channel deposit, there are no solvency guarantees. In these cases, B can accept bonds (as loan requests) or can expect A to top up the channel deposit. For the scenario, where usage of these services are not simultaneous but alternating, it would be ideal to have a way to reallocate channel deposits in a flexible way.

Let us assume A and B both maintain an ordered list of active conditional notes (bonds and bounties) issued by A.
Each time A issues a conditional note or pays for a fulfilled one or one expires, the list is updated.
The hashes of active notes in order of issuance is organised in a swarm tree, the
root hash of which is signed by B and passed to A together with a so called
\gloss{soft channel deposit claim}. This claim constitutes an acknowledgment of
the outstanding liabilities of A towards B. Assume further that there exist a notion of \gloss{epoch},
a fixed settlement period at the end of which B needs to sign off on the total sum of active
outstanding conditional notes issued by A to B.
Formally, a \gloss{soft channel deposit claim} is a note with the following fields:

\begin{itemize}
  \item A's contract address
  \item the total sum of all active outstanding conditional notes
  \item the epoch index
\end{itemize}
