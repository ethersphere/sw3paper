\section{Escrow conditions on future payments}

A promissory note is a redeemable promise of payment potentially maturing in the future. For example a cheque (see section \ref{subsubsec:per-user-guarantees}) is a promise of payment that can be redeemed at any time by `cashing' the cheque. 

We define the general notion of a promissory note as a \emph{promise} that entitles a \emph{beneficiary} to \emph{future payment} under a \emph{condition}. Since any such promise effectively puts the money in escrow, the condition can also be referred to as an  \gloss{escrow condition}. A promissory note can contain the following fields:

  \begin{itemize}
    \item an index or serial number
    \item an amount
    \item a beneficiary
    \item and escrow condition / address
    \item a valid-from time
    \item a valid-until time
    \item a remark
  \end{itemize}

In the simplest case in which no escrow condition or maturation date is given: the note represents a simple cheque as discussed above.
A maturation date for the promissory note (expressed as a blockheight or timestamp) can be specified to indicate the earliest possible occasion a note can be redeemed (valid-from) as well as a deadline when the promise expires (valid-until).
During the validity period, the promise can be redeemed only if the escrow condition is met.


\subsection{Recurring payments}

A cheque represents an adhoc promise to redeem an amount. A transfer of funds is sanctioned by the issuance of a cheque and cheques are issued as needed. There are, however, services where recurring payments are predictable and therefore the installments can be sanctioned in advance. The primary example of this are subscription based services. 

The chequebook contract can be expanded to support such services. Authorising a recurring payment (to a contract) would be achieved by signing a blank cheque (a cheque with an unspecified amount) against the beneficiary possibly with a valid-until date. The beneficiary (a smart contract) would then be able to withdraw payments from the chequebook at will. It would be up to the smart contract code of the beneficiary to ensure that only the `correct' amounts are transferred. 

\subsection{Bonds}

A note with an amount, a beneficiary and a future valid-from date is effectively a bond. The amount represents all outstanding payment obligations that the beneficiary is entitled to once the bond matures. Such a note can either be collateralised (as in section \ref{subsubsec:per-user-guarantees}) or it can come without a guarantee (as in \ref{subsec:simple-chequebook}). When such a note is not collateralised at the time of issuance, accepting the bond is equivalent to granting a loan\footnote{A sends funds to B and B issues an unsecured bond in return}.


\subsection{Escrow conditions and escrow witnesses}

Conditions for payment that are more complex than a validity period are captured by the escrow condition. A promissory note can specify an address of an escrow contract which is to act as the judge, determining whether the escrow conditions have been met.

A note with an escrow field specified is called a \gloss{conditional bond} and can represent a \gloss{service contract} with the escrow condition defining what constitutes the successful delivery of the service. 
If no beneficiary is specified, such note is essentially a \gloss{bounty} - a payment to be made to the first person who can satisfy the escrow condition. 

Thus, in order for payments to be made, the escrow condition must be met. When the escrow condition is to be verified, the owner, the beneficiary address, and the note id (the hash of the signed note) are passed as arguments to the {\gloss{testimonyFor}} method of the escrow contract. By implementing this method the contract conforms to the \gloss{witness} contract  interface (section \ref{sec:courtroom}). Given the witness contract, the escrow condition is implicitly defined as whatever state makes the escrow witness give a positive testimony. 


Figure \ref{fig:taxonomy} summarises the various types of promissory notes.

\newcommand{\tick}{\checkmark}
\newcommand{\opt}{?}
\begin{center}
\begin{figure}
\begin{center}
\begin{tabular}{|l|r||c|c|c|c|c|c|c|}
\hline
note & fields
& index
& amount
& beneficiary
& escrow
& valid-from
& valid-until
& remark
\\
\cline{2-9}

type & type 
& int256
& int256
& address
& address
& int256
& int256
& byte32
\\
\hline
\hline
\multicolumn{2}{|l||}{cheque}   & \tick & \tick & \tick & & & \opt& \opt
\\
\multicolumn{2}{|l||}{authorisation} &  & \tick & \tick & & & \opt& \opt
\\
\multicolumn{2}{|l||}{bond} & \tick & \tick & \tick & & \tick & \opt& \opt
\\
\multicolumn{2}{|l||}{conditional bond} &  & \tick & \tick & \tick & \tick & \opt& \opt
\\
\multicolumn{2}{|l||}{bounty} & \tick &  \tick & & \tick & \tick & \opt& \opt
\\
\multicolumn{2}{|l||}{soft channel deposit} &  \tick & \tick & & & && \opt
\\
\hline
\end{tabular}
\end{center}
\caption{Taxonomy of promissory notes: '\tick' indicates a mandatory field, '?' indicates optional field. Types show the corresponding solidity type to encode in the ABI. }
\label{fig:taxonomy}
\end{figure}
\end{center}


% \subsection{Secure escrow and liquidity}

\subsection{Conditional bonds and Swap: invoice, and cheque}

Let us assume that node A issues a conditional note (or bounty) to B with expiry (valid-until) time T. T represents the earliest time that A can consider the note unfulfilled (the service undelivered), before that we say that the note is \emph{active}. Let us further assume that B fulfills the condition while the note is active.

Invoking the escrow contract directly B can of course redeem the note, however it would be nice if this could be subsumed in the Swap traffic so that no expensive on-chain operation is necessary. 
To redeem the note in Swap, B notifies A of successful delivery of the service and issues  an \gloss{invoice}. This invoice contains the hash of the note and the current cumulative swap balance and last cheque serial number. The invoice thus contains the swap balance at the time of delivery as well as the note that is to be redeemed. It serves to indicate that the subsequent cheque will be the one to pay the outstanding amount for the note.

As a response, A is expected to send a cheque with an updated channel balance reflecting the payment of the invoice - the amount in the conditional note is added to the cumulative total. If A refuses to do this, B can still send the conditional note in a transaction to the escrow contract on-chain. When the witness validates the delivery condition, and the testimony is positive, the amount is escrowed and a grace period starts. During the grace period, A can respond by sending in the appropriate cheque to the contract as proof that the cheque was issued.

\subsection{Soft channel deposits}
A nonmaturing cheque is a deposit locked up as collateral to secure varying balance.
The total value of these deposits serve as guaranteed collateral for any debt and
therefore conceptually equivalent to a channel deposit.
In the following we describe a construct which gives 100$\%$ guarantee of solvency
on outstanding conditional bonds without explicitly assigning deposits to channels on chain.%
%
\footnote{In the on-chain case channel deposits are explicitly stored in the contract, therefore overspending
(using the amount to collateralise multiple creditors) is impossible.}

Let us assume A and B both maintain an ordered list of active conditional notes issued by A.
Each time A issues a conditional note or pays for a fulfilled one or one expires, the list is updated.
The hashes of active notes in order of issuance is organised in a swarm tree, the
root hash of which is signed by B and passed to A together with a so called
\gloss{soft channel deposit claim}. This claim constitutes an acknowledment of
the outstanding liabilities of A towards B. Assume further that there exist a notion of \gloss{epoch},
a fixed settlement period at the end of which B needs to sign off on the total sum of active
outstanding conditional notes issued by A to B.
Formally, soft channel deposit claim is a note with the following fields:

\begin{itemize}
  \item A's contract address
  \item the total sum of all active outstanding conditional notes
  \item the epoch index
\end{itemize}

With each epoch, B signs the note and sends it to A.
Peer A can verify if the soft claim is correct and terminate dealings if not. In such a case soft claim of
the last epoch is taken, whereby A cancels its liability to the conditional notes issued since
the last epoch.
B is incentivised to send the correct soft claim if they want business with A fulfilling their
conditional notes.

After A receives soft channel deposit claims from all peers for the epoch, the claims are
collected in a list (ordered according to the peer index of the beneficiary in A's swap contract).
A signs the swarm hash of the (concatenated) list and sends it alongside the list
in a construct called a \gloss{soft channel deposit allocation table} to each peer.
Upon receiving the soft channel deposit allocation table, B verifies that
(1) the total sum of channel deposits allocated is no greater than the global deposit and
(2) the amount dedicated to B is no less than the sum of all outstanding active
conditional notes issued by A to B.
(3) each active peer in A's swap contract has a corresponding claim for the current epoch and the signature is valid.
This process in illustrated in figure \ref{fig:softchanneldeposit}


\begin{center}
\begin{figure}
\begin{center}
\begin{tikzpicture}
\end{tikzpicture}
\end{center}
\caption{Soft channel deposit}
\label{fig:softchanneldeposit}
\end{figure}
\end{center}




At any point in time, soft channel deposit claims can be consolidated to actual
channel deposits.
The soft channel deposit claim is sent in with the inclusion proof of the claim against the root hash of the
current channel deposit allocation table as a transaction to the
chequebook contract. Upon receiving and validating the claim (signature, resource and epoch verified),
the contract simply reallocates the claimed amount from the global deposit to the channel.
If each peer does this, the global deposit is reduced with the total of soft channel
deposits.%
%
\footnote{To save all peers the trouble of sending inclusion proofs, when A initiates withdrawal from
their global deposit, they need to submit current allocation table starting a grace period during which
creditors are invited to challenge the allocation by presenting a contradicting claim.}

Once the channel deposit (of A towards B) is secured on chain, all the outstanding conditional notes
issued to B are guaranteed to be redeemable in the standard ways.
If a peer is found to issue contradictory channel claims for the same epoch, they relinquish
their right to any claim. For any third party this entails that if a peer's claim is
included in the allocation table, the maximum they are entitled to redeem is limited to:
(1) the sum in the claim if they signed a correct unique claim or
(2) zero if they illegally signed multiple claims for the same epoch.
Futhermore, the allocation table is exhaustive, therefore each
peer signed their claim and no peer without a channel deposit are allowed to have a claim. 
From this, third parties can also conclude that the sum of all claims in the
allocation table is the maximum total deposit that can be consolidated as channel deposits.
In order to secure full liability, it is stipulated in the validity criteria of allocation tables, that the sum of allocations do not exceed the global soft channel deposit.

If the soft channel deposit allocation table is valid,
we can with complete certainty know that the global deposit in A's contract
 could cover all outstanding conditional notes handed to B even if all peers were to redeem all of their outstanding notes (e.g., by satisfying A's conditional bond).
Practically this means then that, after receiving a valid allocation table for the epoch from A,
B has no risk of insolvency when dealing with A's conditional notes.
To insure a completely risk-free flow of conditional notes, the peers only  need to make sure that the total of their outstanding nodes do not exceed the sum of the current soft channel  deposit and the on-chain channel  deposit. 
In fact nodes reallocate soft channel deposits to channels where they suspect that in the following epochs the total outstanding amount from conditional bonds and bounties will increase beyond what the channel deposit can cover and deallocate away from the channel if they are not expected to.
Therefore in a way, hard channel deposits play the role of the payment  threshold in simple swap, whereas soft channel deposits serve are somewhat analogous to cheques or waivers for service-specific collateral against issued service requests. 

In sum, using soft channel deposits enables the chequebook owner to allocate and reallocate funds
as channel deposits in a fairly flexible way without blockchain transaction costs,
yet provide $100\%$ solvency guarantees on all active conditional bonds and bounties to their peers.
