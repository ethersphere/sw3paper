\section{Service guarantees}
\label{sec:courtroom}

In the previous section we introduced tools to compensate service providers.
While rewards are paid out only if delivery is successful, such positive incentivisation may be insufficient if explicit guarantees of future delivery are required.

In this section we describe a system of service contracts and dispute resolution.
Service contracts allow providers (or a set of providers) to offer quality and/or delivery guarantees to customers. First we describe how to give service guarantees (swear), then discuss how they can be enforced (swindle). Finally we link together the three contracts by showing how the swap interacts with swear and swindle.


\subsection{Commitment and litigation}

Without loss of generality, we can conceptualise a generic service as described by a game contract on the blockchain (see later in \ref{sec:swindle}).


Nodes indicate their \gloss{commitment} to providing a service by signing against the game contract either in a transaction or issuing a signed note offchain. By committing to the game contract  providers can be challenged if they don't comply with the rules.

Litigation on the other hand, allows customers to initiate an on-chain court procedure in case they suspect foul play. Such accountability is crucial for the design of scalable decentralised service economies.

We will describe the trial process for dispute resolution and how it evaluates the evidence against the accused provider. The verification is the same as the one used in the context of sanctioning outpayments when the escrow condition is evaluated after service delivery. However, if a provider is proven guilty by the court on the grounds of not complying with the rules specified in the service contract, various punitive measures will be automatically enforced. In most cases, this comprises losing some deposit that provider previously locked up as collateral to back their promise.%
%
\footnote{Instead of or in addition to losing their deposits, nodes would also suffer loss of reputation. If any demonstrated violation of terms is recorded on the public blockchain, users can expect clean histories as a measure of reputation.}
%
By registering on a service contract and putting up a deposit if required, providers \emph{swear} to play by the rules. This deposit is locked up and used later as compensatory insurance transferred to the customer, redistributed or burnt.

\subsection{Trial}
\label{sec:swindle}

If a service provider is suspected of non-compliance, they can be challenged by opening a case with the swear contract analogous to filing a case and have it 
tried in the courtroom.
An abstract trial is composed of stages of litigation, at each stage the court procedure calls witnesses and, depending on their testimony, advances the trial into the next stage. Witnesses can refuse testimony if evidence is not (yet) available in which case the trial is paused until the grace period allowed to submit evidence is over. Eventually the trial concludes with a guilty or not guilty verdict followed by automatically enforced punitive measures in the former case or compensation to the accused in the latter.

The specific service contract describes the rules of the service by assigning particular expert witnesses to trial stages. These witnesses evaluate evidence submitted to them in relation to the case and give testimony if asked by the judge.
The request--response process between the judge and the witnesses can thus be standardized even though each witness may require different data to be submitted to them as evidence.

To implement such an abstract dispute resolution system, the trial is described by a \emph{finite state automaton} the states of which correspond to stages of litigation. The transitions are labelled with various outcomes of evaluating evidence submitted to the respective expert witness associated with the stage (true, false, N/A). A witness is a smart contract that implements the \emph{testimony} function which takes as arguments the provider address, the plaintiff address, and the note id (the hash of the signed note) representing the service request.

The \gloss{swindle} contract orchestrates the transition through the stages by calling the witness contracts, handles grace periods to control the deadline for submitting evidence (challenges or their refutations) to eventually reach a guilty/non-guilty verdict on the grounds of witness testimonies.
In case of a guilty verdict, the deposit sworn on is forfeited and the peer's registration is terminated as executed by the swear contract.

\begin{center}
\begin{figure}
\begin{center}
\begin{tikzpicture}[->,>=stealth',shorten >=1pt,auto,node distance=2c m,
                    semithick]
  \tikzstyle{witness}=[fill=orange!50,draw]
  % \tikzstyle{every state}=[fill=gray!10,draw]

  \node[initial,state] (S)                    {$S$};
  \node[state,witness]         (W0_1) [above right of=S] {$w_0$};
  \node[state,witness]         (W0_2) [below right of=S] {$w_0$};
  \node[state,fill=green!60]         (N/G) [below right of=W0_1] {$N/G$};
  \node[state,witness]         (W1) [above right of=N/G] {$w_1$};
  \node[state,witness]         (W2) [below right of=N/G] {$w_2$};
  \node[state,witness]         (W3) [below right of=W1] {$w_3$};
  \node[state,fill=red!60]         (G) [right of=W3]       {$G$};

  \path (S) edge   node {$A_0$} (W0_1)
            edge   node {$A_1$} (W0_2)
        (W0_1) edge [loop above] node {n/a} (W0_1)
            edge [red,thick]             node {T} (W1)
            edge [green,thick]             node {F} (N/G)
        (W0_2) edge [loop below] node {n/a} (W0_2)
            edge [red,thick]             node {T} (W2)
            edge [green,thick]             node {F} (N/G)
        (W1) edge [loop above] node {n/a} (W1)
            edge [red,thick]             node {T} (W3)
            edge [green,thick]             node {F} (N/G)
        (W2) edge [loop below] node {n/a} (W2)
            edge  [red,thick]            node {T} (W3)
            edge  [green,thick]            node {F} (N/G)
        (N/G) edge              node {*} (S)
        (W3) edge [loop below] node {n/a} (W3)
            edge  [green,thick]            node {F} (N/G)
            edge [red,thick]  node {T} (G);
\end{tikzpicture}
\end{center}
\caption{Trial stages represented as a finite state automaton describing a service contract. Upon receiving an accusation ($A_0$ or $A_1$) in a transaction, the case contract moves to the first litigation state. The orange circles represent states where a witness contract is called, the following state depends on the outcome of the witness testimony (the return value of the $\mathit{testimony}$ function call).
The other nodes call the internal $\mathit{process}$ function: to pay compensation if the challenge is refuted (N/G) or enforce the sentence in case of a guilty verdict (G).}
\label{fig:servicecontract}
\end{figure}
\end{center}

\subsection{Enforcement}

When a guilty verdict is reached, the swindle contract calls back to the swear contract to enforce the sentence, the forfeiture of the game deposit. The swear contract acts as executor of this decree, in a way similar to how a court order is observed by the agencies in a position to enforce sentences.  

The contract that handles the deposits and registers which node plays which game is called \gloss{swear}.
The swear contract is a registry of commitments, a mapping from identities to game contracts.%
%
\footnote{One can imagine not completely open swear registries which curate a whitelist of safe or recommended games. Similarly to an appstore, these contracts could catalogue service games.}
%
An entry is created if a provider signs off on a service game contract in an act of \gloss{commitment}. Essentially, the swear contract only needs to record how much deposit each commitment promised and let the swap channel handle the deposit.
The swear contract can  ask the swap if the provider has insufficient funds dedicated as swear deposits. This is simply implemented by channel deposit allocation: nodes allocate the appropriate amount as channel deposit with the swear contract as beneficiary.%
%
\footnote{If the grace period for withdrawal from hard channel deposits is implemented by a call to the channel beneficiary, then we can handle in an elegant way the cases where deposits are locked as collateral for services with arbitrary validity period.}
%
Commitments are implemented as a signed note containing an amount, the swear contract address as the escrow witness, the games contract address in the remark field, optionally a beneficiary and validity dates. Commitments are a subtype of conditional bonds which when submitted to the swap, call out to the swear contract as its witness and starts a new case against the issuer. 
Interpreted as a service request, a commitment authorises the transfer of a balance up to the amount (actual amount determined by the witness) and asks the swear contract to handle all service complaints against the provider. The swear contract implements the witness interface and when its \emph{testimony} function is called, it prepares the case and authorises the swindle contract to lead the trial. When the guilty or not-guilty verdict is reached, the swindle contract calls back to the swear contract that sets out to execute the sentence with a call to $execute$ giving a case id as argument.

The sentence starts with transferring the deposit amount to the deposit handler contract calling \emph{execute} with the case id.


In this section we connected the dots and showed how swap swear and swindle interact.
The swap contract can validate promissory notes so it can itself serve as witness for the swindle contract.
When a commitment is sent to the swap contract, it calls the swear contract as an escrow witness which in turn opens a case and sanction the swindle contract to lead a trial.
Once the verdict is reached, the sentence is executed by the swear contract which has authorisation to withdraw the deposit from the swap contract that holds it in the form of hard channel deposit. Figure \ref{fig:sw3} gives an  overview of these interactions.

\begin{center}
\begin{figure}
\begin{center}
\begin{tikzpicture}
\end{tikzpicture}
\end{center}
\caption{Swap, Swear and Swindle contracts and their interactions:
}
\label{fig:sw3}
\end{figure}
\end{center}