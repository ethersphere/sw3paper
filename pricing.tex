
The problem of pricing appears already when we introduce indirect transactions in a service provision network. When an intermediate relaying node takes over a bounty note for a task, it is implicitly assumed that they can find a node they forward it to. This presupposes two things, first that there exists a node sworn on the same service provision game and second, that the node taking over the task is willing to do so for a price not higher than the upstream bounty promised to give for it. The first criterion is guaranteed by GRANT (in particular a healthy kademlia topology), but the second needs to be stipulated. The easiest way to achieve it is to have fixed maximum price for a task set centrally for the entire service network. 

In order to allow the market to adjust the unit price, we will propose two distinct mechanisms that model a market-driven pricing scheme: one that decreases the price and the other that increases depending on demand and supply.





Let us assume that there is a notion of epoch, a period where price changes are set. Issuers and their peers know the fix maximum price of current epoch and can enforce that valid bounties specify this amount. 
Assume further that we modify the task allocation scheme so that for each task there is more than one node potential service providers that are granted the opportunity to service a task. Only one or a subset of them will actually get rewarded for doing so. 
This means that some nodes at least some of the time will do spurious work. If the number of nodes granted the task is on-average constant, and opportunities to do service is uniform then uncompensated work is  predictable and can be factored in the price. 
If the bounty specifies a preference which providers get rewarded, 
the potential task providers can compete for the bounty.

An obvious choice for the selection rule for such a competitive bounty is a preference for lower price. 
