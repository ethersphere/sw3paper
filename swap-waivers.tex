\subsection{Waivers}
\label{subsubsec:waivingdebt}

If the imbalance in the swap channel is the result of high variance as opposed to unequal consumption, after a period of accumulating cheques the channel balance starts tilting the other way. Normally it is now up to the other party to issue cheques to its peer resulting in uncashed cheques accumulating on both sides.
To allow for further savings in transaction costs, it could be desirable to be able to `play the cheques off against each other'.

Such a process is possible, but it requires certain deep changes within the chequebook contract. In particular, cashing cheques can no longer be immediate and must incur a security delay, familiar from payment channels.

Let us imagine a system analogous to cheques being returned to the issuer.  
Assume peer A issued cheques to B and the balance was brought back to zero. Later the balance tilts in A's favour but the cheques from A to B have not been cashed. In the real world, user B could simply return the last cheque back to A or provably destroy it. In our case it is not so simple; we need some other mechanism by which B commits not to cash that particular cheque. Such a commitment could take several forms; it could be implemented by B signing a message allowing A to issue a new `last cheque` which has a lower cumulative total amount than before, or perhaps B can issue some kind of `negative` cheque for A's chequebook that would have the effect as if a cheque with the same amount had been paid. 

What all the implementations have in common, is that the chequebook can no longer allow instantaneous cashing of cheques. Upon receiving a cheque cashing request, the contract must wait to allow the other party in question to submit potentially missing information about cancelled cheques or reduced totals. 

We describe one possible implementation below.

To accommodate (semi-)bidirectional payments using a single chequebook we make the following modifications

\begin{enumerate}
    \item All cheques from user A to user B must contain a serial number.
    \item Each new cheque issued by A to B must increase the serial number.
    \item A's chequebook contract records the serial number of the last cheque that B cashed.
    \item During the cashing delay, valid cheques with higher serial number supersede any previously submitted cheques regardless of their face value.
    \item Any submitted cheque which decreases the payout of the previously submitted cheque is only valid if it is signed by the beneficiary.
\end{enumerate}

With these rules in place it is easy to see how cheque cancellation would work. 

% Waivers diagram 
% Author: Fabio Barone, based on Pascal Seppecher taken from http://www.texample.net/tikz/examples/sequence-diagram/

% Agents
\def\A{A}
\def\B{B}
\def\FaceVal{Amount due}
\def\Contract{Contract value}
\def\Totals{Cumulative Sum}

% Message Flows
\def\ChequeA{Cheque $c0$}
\def\ChequeB{Cheque $c1$}
\def\ChequeC{Cheque $c2$}
\def\RedeemC{Redeem Cheque $c1$}
\def\ChequeD{Cheque $c3$}
\def\WFailCumTotTooHigh{Waive attempt $w_4$}
\def\WaiverA{Waive $w_4$}
\def\RedeemFailWrongSN{Redeem $c3$}
\def\WaiverB{Waive $w_5$}
\def\ChequeE{Cheque $c6$}
\def\ChequeF{Cheque $c7$}
\def\RedeemF{Redeem $c7$}
\def\ChequeG{Cheque $c8$}
\def\RedeemG{Redeem $c8$}

% Legend 
%\def\LegendOnChain{On-chain}
%\def\LegendOffChain{Off-chain}




\begin{figure}
\centering


\begin{tikzpicture}[every node/.style={font=\small,
  minimum height=.35cm,minimum width=.5cm}]
 \useasboundingbox (-1cm,-7cm) rectangle (3.5cm,7cm);
   \scope[transform canvas={scale=.75}]


%
% Matrix
\node [matrix, very thin,column sep=2.0cm,row sep=0.4cm] (matrix) at (0,0) {
  &  & \node(0,0) (\A) {}; &                        & \node(0,0) (\B) {};        & \node(0,0) {Serial};               & \node(0,0) (\FaceVal) {};  & \node(0,0) (\Contract) {}; & \node(0,0) (\Totals) {}; \\
  &  & & & & & & &  \\
  &  & & & & & & &  \\
  &  & & & & & & & \\
  &  & \node(0,0) (\A 1) {}; & \node(0,0) (\ChequeA) {}; & \node(0,0) (\B 1) {};& \node(0,0) {0};                    & \node(0,0) {6};          & \node(0,0) {6};            & \node(0,0) {6}; \\
  &  & & & & & & & \\
  &  & \node(0,0) (\A 2) {}; & \node(0,0) (\ChequeB) {}; & \node(0,0) (\B 2) {};& \node(0,0) {1};                     & \node(0,0) {9};          & \node(0,0) {15};            & \node(0,0) {15}; \\ 
  &  & & & & & & & \\
  &  & \node(0,0) (\A 3) {}; & \node(0,0) (\ChequeC) {}; & \node(0,0) (\B 3) {};& \node(0,0) {2};                     & \node(0,0) {13};          & \node(0,0) {28};            & \node(0,0) {28}; \\
  &  & & & & & & & \\
  &  & \node(0,0) (\A 4) {}; & \node(0,0) (\RedeemC) {}; & \node(0,0) (\B 4) {};& \node(0,0) {1};                 & \node(0,0) {15};          & \node(0,0) {19};            & \node(0,0) {28}; \\
  &  & & & & & & & \\
  &  & \node(0,0) (\A 5) {}; & \node(0,0) (\ChequeD) {}; & \node(0,0) (\B 5) {};& \node(0,0) {3};                     & \node(0,0) {14};          & \node(0,0) {33};            & \node(0,0) {42}; \\
  & & & & & & & & \\
  & \node(0,0) {Fails as exceeds Contract value} ;  & \node(0,0) (\A 6) {}; & \node(0,0) (\WFailCumTotTooHigh) {}; & \node(0,0) (\B 6) {}; & \node(0,0) {4};    & \node(0,0) {100};          & \node(0,0) {33};            & \node(0,0) {42}; \\
  &  & & & & & & & \\
  &  & \node(0,0) (\A 7) {}; & \node(0,0) (\WaiverA) {}; & \node(0,0) (\B 7) {};& \node(0,0) {4};                     & \node(0,0) {7};          & \node(0,0) {26};            & \node(0,0) {42}; \\
  &  & & & & & & & \\
  & \node(0,0) {Fails due to wrong s/n}; & \node(0,0) (\A 8) {}; & \node(0,0) (\RedeemFailWrongSN) {}; & \node(0,0) (\B 8) {};& \node(0,0) {3};      & \node(0,0) {42};          & \node(0,0) {26};            & \node(0,0) {42}; \\
  &  & & & & & & & \\
  &  & \node(0,0) (\A 9) {}; & \node(0,0) (\WaiverB) {}; & \node(0,0) (\B 9) {};& \node(0,0) {5};                     & \node(0,0) {26};          & \node(0,0) {0};            & \node(0,0) {42}; \\
  &  & & & & & & & \\
  &  & \node(0,0) (\A 10) {}; & \node(0,0) (\ChequeE) {}; & \node(0,0) (\B 10) {};& \node(0,0) {6};                     & \node(0,0) {11};          & \node(0,0) {11};            & \node(0,0) {53}; \\
  &  & & & & & & & \\
  &  & \node(0,0) (\A 11) {}; & \node(0,0) (\ChequeF) {}; & \node(0,0) (\B 11) {};& \node(0,0) {7};                     & \node(0,0) {15};          & \node(0,0) {26};            & \node(0,0) {68}; \\
  &  & & & & & & & \\
  &  & \node(0,0) (\A 12) {}; & \node(0,0) (\RedeemF) {}; & \node(0,0) (\B 12) {};& \node(0,0) {7};               & \node(0,0) {68};          & \node(0,0) {0};            & \node(0,0) {68}; \\
  &  & & & & & & & \\
  & \node(0,0) {Exchange in other direction ok};   & \node(0,0) (\A 13) {}; & \node(0,0) (\ChequeG) {}; & \node(0,0) (\B 13) {};& \node(0,0) {0};                     & \node(0,0) {};          & \node(0,0) {};            & \node(0,0) {}; \\
  &  & & & & & & & \\
  &  & \node(0,0) (\A 14) {}; & \node(0,0) (\RedeemG) {}; & \node(0,0) (\B 14) {}; & \node(0,0) {1};              & \node(0,0) {};          & \node(0,0) {};            & \node(0,0) {}; \\
  &  & & & & & & & \\
  &  & \node(0,0) (\A 15) {}; &                          ; & \node(0,0) (\B 15) {};&                               &                           &                             &                  \\
};

% Agents labels
\fill 
	(\A) node[draw,fill=white] {\A}
	(\B) node[draw,fill=white] {\B}
	(\FaceVal) node[] {\FaceVal}
	(\Contract) node[] {\Contract}
	(\Totals) node[] {\Totals};

\draw [dashed] 
  (\A) -- (\A 15)
  (\B) -- (\B 15);

% Horizontal flows 
\draw [-{Latex[length=1.5mm,width=2.5mm]}] (\A 1) -- (\B 1);
\draw [-{Latex[length=1.5mm,width=2.5mm]}] (\A 2) -- (\B 2);
\draw [-{Latex[length=1.5mm,width=2.5mm]}] (\A 3) -- (\B 3);
\draw [-{Latex[length=1.5mm,width=2.5mm]}] (\B 4) -- (\A 4);
\draw [-{Latex[length=1.5mm,width=2.5mm]}] (\A 5) -- (\B 5);
\draw [-{Latex[length=1.5mm,width=2.5mm]},color=red] (\B 6) -- (\A 6);
\draw [-{Latex[length=1.5mm,width=2.5mm]}] (\B 7) -- (\A 7);
\draw [-{Latex[length=1.5mm,width=2.5mm]},color=red] (\B 8) -- (\A 8);
\draw [-{Latex[length=1.5mm,width=2.5mm]}] (\B 9) -- (\A 9);
\draw [-{Latex[length=1.5mm,width=2.5mm]}] (\A 10) -- (\B 10);
\draw [-{Latex[length=1.5mm,width=2.5mm]}] (\A 11) -- (\B 11);
\draw [-{Latex[length=1.5mm,width=2.5mm]}] (\B 12) -- (\A 12);
\draw [-{Latex[length=1.5mm,width=2.5mm]}] (\A 13) -- (\B 13);
\draw [-{Latex[length=1.5mm,width=2.5mm]}] (\B 14) -- (\A 14);

% Flows Labels 
\fill
  (\ChequeA) 
    node[above] {\ChequeA}
  (\ChequeB) 
    node[above] {\ChequeB}
  (\ChequeC) 
    node[above] {\ChequeC}
  (\RedeemC) 
    node[above] {\RedeemC}
  (\ChequeD) 
    node[above] {\ChequeD}
  (\WFailCumTotTooHigh) 
    node[above] {\WFailCumTotTooHigh}
  (\WaiverA) 
    node[above] {\WaiverA}
  (\RedeemFailWrongSN) 
    node[above] {\RedeemFailWrongSN}
  (\WaiverB) 
    node[above] {\WaiverB}
  (\ChequeE) 
    node[above] {\ChequeE}
  (\ChequeF) 
    node[above] {\ChequeF}
  (\RedeemF) 
    node[above] {\RedeemF}
  (\ChequeG) 
    node[above] {\ChequeG}
  (\RedeemG) 
    node[above] {\RedeemG};

% Interaction points 
\draw 
  (\A 1) node[minimum size=0.25cm, draw,circle,fill=red!20] {}
  (\B 1) node[minimum size=0.25cm, draw,circle,fill=red!20] {}
  (\A 2) node[minimum size=0.25cm, draw,circle,fill=red!20] {}
  (\B 2) node[minimum size=0.25cm, draw,circle,fill=red!20] {}
  (\A 3) node[minimum size=0.25cm, draw,circle,fill=red!20] {}
  (\B 3) node[minimum size=0.25cm, draw,circle,fill=red!20] {}
  (\A 4) node[minimum size=0.25cm, draw,circle,fill=red!20] {}
  (\B 4) node[minimum size=0.25cm, draw,circle,fill=red!20] {}
  (\A 5) node[minimum size=0.25cm, draw,circle,fill=red!20] {}
  (\B 5) node[minimum size=0.25cm, draw,circle,fill=red!20] {}
  (\A 6) node[minimum size=0.25cm, draw,circle,fill=red!20] {}
  (\B 6) node[minimum size=0.25cm, draw,circle,fill=red!20] {}
  (\A 7) node[minimum size=0.25cm, draw,circle,fill=red!20] {}
  (\B 7) node[minimum size=0.25cm, draw,circle,fill=red!20] {}
  (\A 8) node[minimum size=0.25cm, draw,circle,fill=red!20] {}
  (\B 8) node[minimum size=0.25cm, draw,circle,fill=red!20] {}
  (\A 9) node[minimum size=0.25cm, draw,circle,fill=red!20] {}
  (\B 9) node[minimum size=0.25cm, draw,circle,fill=red!20] {}
  (\A 10) node[minimum size=0.25cm, draw,circle,fill=red!20] {}
  (\B 10) node[minimum size=0.25cm, draw,circle,fill=red!20] {}
  (\A 11) node[minimum size=0.25cm, draw,circle,fill=red!20] {}
  (\B 11) node[minimum size=0.25cm, draw,circle,fill=red!20] {}
  (\A 12) node[minimum size=0.25cm, draw,circle,fill=red!20] {}
  (\B 12) node[minimum size=0.25cm, draw,circle,fill=red!20] {}
  (\A 13) node[minimum size=0.25cm, draw,circle,fill=red!20] {}
  (\B 13) node[minimum size=0.25cm, draw,circle,fill=red!20] {}
  (\A 14) node[minimum size=0.25cm, draw,circle,fill=red!20] {}
  (\B 14) node[minimum size=0.25cm, draw,circle,fill=red!20] {};

% Vertical lifelines
\endscope
\end{tikzpicture}


\caption{Example sequence of mixed cheques and waivers exchange}
\label{fig:waivers-diagram}
\end{figure}


Suppose user A has issued cheques $c_0 \ldots c_n$ with cumulative totals $t_0 \ldots t_n$ to user B. Suppose that the last cheque B cashed was $c_i$. The chequebook contract has recorded that B has received a payout of $t_i$ and that the last cheque cashed had serial number $i$.

Let us further suppose that the balance starts tilting in A's favour by some amount $x$. If B had already cashed cheque $c_n$, then B would now have to issue a cheque of her own using B's chequebook as the source and naming A as the beneficiary. However, since cheques $c_{i+1} \ldots c_n$  are uncashed, B can instead send to A a cheque with A's chequebook as the source, B as the beneficiary, with serial number $n+1$ and cumulative total $t_{n+1} = t_n - x$. Due to the rules enumerated above, A will accept this as equivalent to a payment of amount $x$ by B.  In this scenario, instead of sending a cheque to A, B waive part of their earlier entitlement. This justifies swap as \emph{send waiver as payment}.

This process can be repeated multiple times until the cumulative total is brought back to $t_i$. At this point all outstanding debt has effectively been cancelled and any further payments must be made in the form of a proper cheque from B's chequebook to A (See figure \ref{fig:waivers-diagram}).



% 
% \begin{center}
% \begin{figure}
% \begin{center}
% \begin{tabular}{ccc}
% \begin{tikzpicture}
% \node (middle)[draw, rectangle, fill=orange!50, minimum height=1em, minimum width=4em]{};
% \node (leftred) [draw, rectangle, fill=red!70, minimum height=1em, minimum width=1em, node distance=2.5em, left of=middle]{};
% \node (rightred)[draw, rectangle, fill=red!70, minimum height=1em, minimum width=1em, node distance=2.5em, right of=middle]{};
% \node (zero) [above of=middle,node distance=2em, text height=1em] {};
% \node (zerod) [below of=zero, node distance=3.5em] {};
% \node (balance) [left of=zero,node distance=2em, text height=1.5em] {$-n$};
% \node (balanced) [below of=balance,node distance=3.5em] {};
% \draw [dashed](zero)--(zerod);
% \draw [very thick](balance)--(balanced);
% \node (payment) [below of=zerod, node distance=1em]{};
% \node (cheque) [draw, left of=payment, node distance=1.7em, minimum height=1em, minimum width=3.4em, fill=blue!30, rectangle]{$C+n$};
% 
% \node (swap) [left of=leftred,minimum width=1em,align=right]{swap};
% \node (cheques) [below of=swap,minimum width=1em, node distance= 2.5em, align=right]{cheques};
% \node (waivers) [below of=cheques,minimum width=1em, node distance= 2em, align=right]{waivers};
% \end{tikzpicture}
% &
% \begin{tabular}{c}
%   $\Longrightarrow$
% \\ \\ \\ \\
% \end{tabular}
% &
% \begin{tikzpicture}
% \node (middle)[draw, rectangle, fill=orange!50, minimum height=1em, minimum width=4em]{};
% \node (leftred) [draw, rectangle, fill=red!70, minimum height=1em, minimum width=1em, node distance=2.5em, left of=middle]{};
% \node (rightred)[draw, rectangle, fill=red!70, minimum height=1em, minimum width=1em, node distance=2.5em, right of=middle]{};
% \node (zero) [above of=middle,node distance=2em, text height=1.5em] {$0$};
% \node (zerod) [below of=zero, node distance=3.5em] {};
% % \draw [dashed](zero)--(zerod);
% \draw [very thick](zero)--(zerod);
% \node (payment) [below of=zerod, node distance=1em]{};
% \node (cheque) [draw, left of=payment, node distance=1.7em, minimum height=1em, minimum width=3.4em, fill=blue!30, rectangle]{$C+n$};
% \node (waivers) [below of=payment, node distance=2em]{};
% \node (waiver) [right of=waivers,minimum width=2em, node distance=1em,rectangle,draw,fill=purple!50]{$n$};
% \end{tikzpicture}
% \end{tabular}
% \end{center}
% 
% \caption{Peer A's swap balance (with respect to B) reaches the payment threshold $n$ (left),
% A sends a waiver to peer B. B keeps the waiver and restores the swap balance to zero}
% \label{fig:waiverswap}
% \end{figure}
% \end{center}
% 
